%	---------------------------------	%
%		Layout Einstellungen				%
%	---------------------------------	%

% Umlaute unter UTF8 nutzen
\usepackage[utf8]{inputenc}

% deutsche Sonderzeichen und Silbentrennung
\usepackage[ngerman]{babel, translator}

%  Schriftart Times
\usepackage{mathptmx}
\usepackage[scaled=.90]{helvet}
\usepackage{courier}

% Zeichenencoding
\usepackage[T1]{fontenc}
\usepackage{fix-cm}


% Paket für Seitenrandabstände und Einstellung für Seitenränder
\usepackage{geometry}
\geometry{
	left=2.5cm,
	right=2.5cm, 
	top=2.5cm, 
	bottom=3cm
}

% Paragraph styles
\setlength{\parindent}{0cm}
\setlength{\parskip}{6pt}

% Schaltet den zusätzlichen Zwischenraum ab, den LaTeX normalerweise nach einem Satzzeichen einfügt.
\frenchspacing

% Gliederungstiefe einstellen
\setcounter{secnumdepth}{5}
\setcounter{tocdepth}{5}

 
% Kopf- und Fußzeile %

\usepackage{fancyhdr} %Paket laden
\pagestyle{fancy} %eigener Seitenstil
\fancyhf{} %alle Kopf- und Fußzeilenfelder bereinigen
\fancyhead[L]{\nouppercase{\leftmark}} %Kopfzeile links
\fancyhead[C]{} %zentrierte Kopfzeile
\fancyhead[R]{} %Kopfzeile rechts
\renewcommand{\headrulewidth}{0.5pt} %obere Trennlinie
\fancyfoot[R]{\thepage} %Seitennummer
\fancypagestyle{plain}{}
%\renewcommand{\footrulewidth}{0.4pt} %untere Trennlinie

% Abstand Kopf- und Fußzeile vom Rand
\setlength{\headheight}{1.25cm}
\setlength{\footskip}{1cm}

% Abstand Text zur Kopfzeile
\setlength{\headsep}{1cm}


% bricht lange URLs "schoen" um
\usepackage[hyphens,obeyspaces,spaces]{url}

% erzeugt Inhaltsverzeichnis mit Querverweisen zu den Kapiteln (PDF Version)



% Disable single lines at the start of a paragraph (Schusterjungen)
\clubpenalty = 10000

% Disable single lines at the end of a paragraph (Hurenkinder)
\widowpenalty = 10000
\displaywidowpenalty = 10000

% zum Anzeigen des Seitenlayouts
%\usepackage{showframe}


%	---------------------------------	%
%				Pakete					%
%	---------------------------------	%

\usepackage{graphicx}% Grafiken aus PNG Dateien einbinden
\usepackage[onehalfspacing]{setspace}%-- Zeilenabstand 1,5
\usepackage{colortbl}%Einfärben von Tabellen-Zellen, Zeilen, Spalten
\usepackage{tocloft}% Abbildungs- und Tabellenbeschriftung ändern
\usepackage{array}% für Tabellen
\usepackage{multicol}% Paket für multirow Befehl (Tabellen)
\usepackage{multirow}% Paket für multirow Befehl (Tabellen)
\usepackage{longtable}% mehrseitige Tabellen ermöglichen
\usepackage{floatflt}% floatende Bilder ermöglichen
\usepackage{float}
\usepackage{amsmath}% Mathematik Paket
\usepackage{amssymb}% Mathematische Symbole importieren
\usepackage{amsthm}
\usepackage{amsbsy}
\usepackage{textcomp}
\usepackage{upgreek}
\usepackage{wrapfig}
\usepackage{subcaption}
\usepackage{pdfpages}% einbinden von PDF-Dateien
\usepackage{lastpage}
\usepackage{caption}% Paket für Beschriftungen
%\usepackage{ulem}% doppelt unterstreichen \uuline
\usepackage[normalem]{ulem}% doppelt unterstreichen \uuline
\usepackage{tabularx}
\usepackage{lscape}
\usepackage{xcolor}
\usepackage{siunitx}% Paket für SI-Einheiten
\usepackage[right]{eurosym}% Eurozeichen einbinden
\usepackage{color}% Paket für Textfarben
\usepackage{fancybox}% Paket für Boxen im Text
\usepackage{setspace}% Paket für Zeilenabstand
\usepackage{capt-of}% Bildbezeichner
\usepackage[autostyle=true,german=quotes]{csquotes}% Paket für Anführungszeichen
\sisetup{locale = DE}
%\usepackage{hyperref}% Paket für Referenzen
\usepackage[
   colorlinks,        % Links ohne Umrandungen in zu wählender Farbe
   linkcolor=black,   % Farbe interner Verweise
   filecolor=black,   % Farbe externer Verweise
   citecolor=black    % Farbe von Zitaten
]{hyperref}
\usepackage{makeidx}% Stichwortverzeichnis
\usepackage{nomencl}% Symbolverzeichnis
\makenomenclature

%	---------------------------------	%
%			Tabellen und Bilder			%
%	---------------------------------	%

% Rahmen in Textbreite um Objekte
\floatstyle{boxed}
\restylefloat{figure} % Abbildungen
%\restylefloat{table} % Tabellen
\restylefloat{listing} % Codebeispiele

% Captions linksbündig orientieren
\captionsetup{justification=raggedright,singlelinecheck=false}

% Anpassungen Tabellenspalten
% bei fester Ausrichtung Spaltenbreite vorgeben
\newcolumntype{L}[1]{>{\raggedright\arraybackslash}p{#1}}
\newcolumntype{C}[1]{>{\centering\arraybackslash}p{#1}}
\newcolumntype{R}[1]{>{\raggedleft\arraybackslash}p{#1}}

% Anpassung der Tabellenspaltenhöhe
\renewcommand{\arraystretch}{1.2}

% Anpassen der captions
\addto{\captionsngerman}{
\renewcommand*{\figurename}{Abb.}
\renewcommand*{\tablename}{Tab.}
}

% Anpassen der Nummerierung
\numberwithin{equation}{section}

% Tabellenfarben

% Farbdefinitione
%\definecolor{dunkelgrau}{rgb}{0.8,0.8,0.8}
%\definecolor{hellgrau}{rgb}{0.95,0.95,0.95}

%	---------------------------------	%
%			 Codeverzeichnis				%
%	---------------------------------	%

\usepackage{listings}% Paket für Codebeispiele

\lstset{
language=Matlab,
basicstyle=\small\ttfamily\color{black},
commentstyle = \ttfamily\color{green},
keywordstyle=\ttfamily\color{blue},
stringstyle=\color{orange},
backgroundcolor=\color{white},
frame=single,
showstringspaces=false,
captionpos=b,
}

\makeatletter
\begingroup\let\newcounter\@gobble\let\setcounter\@gobbletwo
  \globaldefs\@ne \let\c@loldepth\@ne
  \newlistof{listings}{lol}{\lstlistlistingname}
\endgroup
\let\l@lstlisting\l@listings
\makeatother

\setlength{\cftlistingsindent}{0cm}
\renewcommand*{\cftlistingspresnum}{\lstlistingname~}
\settowidth{\cftlistingsnumwidth}{\cftlistingspresnum}
\addtolength{\cftlistingsnumwidth}{2cm}
\renewcommand{\cftlistingsaftersnum}{:}

% Anpassen der captions
\addto{\captionsngerman}{
\renewcommand*{\lstlistingname}{Code}
}

%	---------------------------------	%
%		Glossar & Symbolverzeichnis		%
%	---------------------------------	%

% Symbolverzeichnis
\newcommand{\nomunit}[1]{%
\renewcommand{\nomentryend}{\hspace*{\fill}#1}}

% Glossar
\usepackage[toc,nonumberlist,acronyms,shortcuts,translate=babel]{glossaries}
\renewcommand*{\glspostdescription}{}
\makeglossaries
\loadglsentries{Dateien/Verzeichnisse/Glossar.tex}



%	---------------------------------	%
%		Literaturverzeichnis				%
%	---------------------------------	%

% Erscheinungsbild
\bibliographystyle{alphadin}